\documentclass[twocolumn,landscape,12pt,paper=a4,DIV=15]{ling-handout}

\usepackage{kantlipsum}

\addbibresource[location=remote]{/home/patrl/GitHub/bibliography/elliott_mybib.bib}

\title{\texttt{ling-handout} example}

\author{Patrick D.\,Elliott}

 \begin{document}

\maketitle

\section{Introduction}

\begin{frame}{Background}

  In D(istributed)M(orphology) (\citealt{halle1993,halle2000}, etc.) properties
  which were traditionally the preserve of individual lexical items are
  distributed across different components of the grammar.

  Their are no categorised lexical items in DM. Formatives - the atoms of
  computation - consist of syntactic feature bundles and \emph{roots} (labelled
  \dmroot{}). Roots are the locus of idiosyncrasy in the grammar.
  %
  \begin{center}
    \scalebox{0.8}{
    \begin{forest}
      [{n}
        [{n}]
        [{\dmroot{dog}}]
      ]
    \end{forest}
    %
    \hspace{3em}
    %
    \begin{forest}
      [{v}
        [{v}]
        [{\dmroot{dog}}]
      ]
    \end{forest}
    %
    \hspace{3em}
    %
    \begin{forest}
      [{v}
      [{v}]
      [{...}
        [{\sc pl}]
        [{\dmroot{person}}]
      ]]
    \end{forest}
  }
  %
\end{center}
%
    Forms and meanings are assigned to leaves of the tree via post-syntactic
    spellout rules.

\end{frame}

\begin{frame}{Overview}

  \begin{block}{Question at center of this talk:}
    \[
      \evaluation{\dmroot{}} = ???
    \]
  \end{block}

  \begin{block}{Answer suggested here:}

    \begin{itemize}

      \item Roots are never argument-taking; they simply denote predicates of type \type{\langle e,t
          \rangle}

      \item $D_{\type{e}}$ includes not only individuals but also events,
        times, etc. Categorizers are/can be responsible for further restriction.

      \item An intuitive way of thinking about this: roots are polymorphic
        predicates; categorizers are responsible for typing.

    \end{itemize}

  \end{block}

\end{frame}

\begin{frame}

  \begin{block}{Empirical focus:}
    Varieties of nominalizations, and nominalization of clausal-embedding verbs.
  \end{block}

  \metroset{block=transparent}

  \begin{block}{Roadmap}
    \begin{itemize}
    \item \citeposs{grimshaw_argument_1990} partitioning of
      nominalizations into Complex Event Nominals, Simple Event Nominals, and
      Result Nominals.
      \item \citeposs{moulton2014} analysis of nominalizations in terms of
        argument-taking roots.
      \item Clausal-embedding verbs and (the unavailability of) Complex Event
        Nominals.
      \item \citeauthor{elliott2016}'s (\citeyear{elliott2016,elliott2017_sub})
        account of clausal embedding.
      \item Analysis of nominalizations with roots as predicates.
    \end{itemize}
  \end{block}

\end{frame}

% \begin{frame}{References}

\printbibliography

% \end{frame}

\end{document}

%%% Local Variables:
%%% mode: latex
%%% TeX-master: t
%%% TeX-command-extra-options: "-shell-escape"
%%% End:
